%------------------------
% Resume Template
% Author : Anubhav Singh
% Github : https://github.com/xprilion
% License : MIT
%------------------------

\documentclass[a4paper,20pt]{article}

\usepackage{latexsym}
%% \usepackage[empty]{fullpage}
\usepackage{titlesec}
\usepackage{marvosym}
\usepackage[usenames,dvipsnames]{color}
\usepackage{verbatim}
\usepackage{enumitem}
\usepackage[pdftex]{hyperref}
\usepackage{fancyhdr}

\pagestyle{fancy}
\fancyhf{} % clear all header and footer fields
\fancyfoot{}
\renewcommand{\headrulewidth}{0pt}
\renewcommand{\footrulewidth}{0pt}

% Adjust margins
\addtolength{\oddsidemargin}{-0.530in}
\addtolength{\evensidemargin}{-0.375in}
\addtolength{\textwidth}{1in}
\addtolength{\topmargin}{-.45in}
\addtolength{\textheight}{1in}

\urlstyle{rm}

\raggedbottom
\raggedright
\setlength{\tabcolsep}{0in}

% Sections formatting
\titleformat{\section}{
  \vspace{-10pt}\scshape\raggedright\large
}{}{0em}{}[\color{black}\titlerule \vspace{-6pt}]

%-------------------------
% Custom commands
\newcommand{\resumeItem}[2]{
  \item\small{
    \textrm{#1}{: #2 \vspace{-2pt}}
  }
}

\newcommand{\resumeItemWithoutTitle}[1]{
  \item\small{
    {\vspace{-2pt}}
  }
}

\newcommand{\resumeSubheading}[4]{
  \vspace{-1pt}\item
    \begin{tabular*}{0.97\textwidth}{l@{\extracolsep{\fill}}r}
      \textbf{#1} & #2 \\
      \textit{#3} & \textit{#4} \\
    \end{tabular*}\vspace{-5pt}
}


\newcommand{\resumeSubItem}[2]{\resumeItem{#1}{#2}\vspace{-3pt}}

\renewcommand{\labelitemii}{$\circ$}

\newcommand{\resumeSubHeadingListStart}{\begin{itemize}[leftmargin=*]}
\newcommand{\resumeSubHeadingListEnd}{\end{itemize}}
\newcommand{\resumeItemListStart}{\begin{itemize}}
\newcommand{\resumeItemListEnd}{\end{itemize}\vspace{-5pt}}

%-----------------------------
%%%%%%  CV STARTS HERE  %%%%%%

\begin{document}

%----------HEADING-----------------
\begin{tabular*}{\textwidth}{l@{\extracolsep{\fill}}r}
  \textbf{{\LARGE Zhi Chen}} & Email: \href{mailto:zhi.chen.3@stonybrook.edu}{zhi.chen.3@stonybrook.edu}\\
  \href{https://github.com/zhichen3}{Github: ~~github.com/zhichen3}  & ORCID:~~~0000-0002-2839-107X \\
% -----Personal-Website-----------
  %\href{https://zhichen3.github.io/}{Website: zhichen3.github.io/} \\
%---------------------------------
\end{tabular*}

%-----------EDUCATION-----------------
\section{~~Education}
    \resumeSubHeadingListStart
        \resumeSubheading
            {Stony Brook University}{Stony Brook, NY}
            {Doctor of Philosophy - Physics;}{Aug 2023 - Present}
        \resumeSubheading
            {Stony Brook University}{Stony Brook, NY}
            {Master of Arts - Physics; GPA: 3.94/4.00}{Aug 2021 - May 2023}
            %{\scriptsize \textit{ \footnotesize{\newline{}\textbf{Courses:} Stars \& Radiation, Cosmology, Galaxies, Classical Mechanics, Quantum Mechanics I, Computational Methods}}}
        \resumeSubheading
            {Stony Brook University}{Stony Brook, NY}
            {Bachelor of Science - Physics;  GPA: 3.96/4.00; Summa Cum Laude}{Aug 2017 - May 2021}
            %{\scriptsize \textit{ \footnotesize{\newline{}\textbf{Courses:} Stars \& Radiation, Cosmology, Galaxies, General Relativity, Classical Mechanics, E\&M, Statistical Mechanics, Quantum Mechanics}}}
  \resumeSubHeadingListEnd
	    
\vspace{-5pt}

\section{Research Experience}
  \resumeSubHeadingListStart
    \resumeSubheading
        {PhD Thesis Research/ Research Assistant}{}
        {Graduate Student Research}{Aug 2023 - Present}
        
\vspace{-2pt}

    \resumeSubheading
        {Master Thesis Research / Research Assistant}{}
        {Graduate Student Research}{Jan 2022 - Aug 2023}
		\resumeItemListStart
            \resumeItem{Primary Project Description}
            {Investigated how different reaction networks, screening methods, and integration methods affect thermonuclear flame propagation for X-Ray Bursts using \href{https://github.com/AMReX-Astro/Castro}{\emph{Castro}}, an adaptive mesh, astrophysical radiation hydrodynamics simulation code.}
            \resumeItem{Astronomy Software Development}{Worked collaboratively in a team to develop astronomy software called \href{https://github.com/python-hydro/pyro2}{\emph{pyro2}}, \href{https://github.com/pynucastro/pynucastro}{\emph{Pynucastro}}, \href{https://github.com/AMReX-Astro/Microphysics}{\emph{Microphysics}}, and \href{https://github.com/AMReX-Astro/Castro}{\emph{Castro}}.}
            %The former is a Python library for interactively creating and exploring nuclear reaction networks, and the latter contains astrophysical microphysics routines for stellar explosions using C++.}
		\resumeItemListEnd		
\vspace{-2pt}


    \resumeSubheading
        {SULI Program at Brookhaven National Lab + Continued Research}{}%{Remote}
        {Summer Internship + Student Research}{Jun 2021 - June 2022}
		\resumeItemListStart
            \resumeItem{General Setting}
            {Worked in a small group developing a technique called two-photon interferometry, which has the potential of measuring the relative separation of two luminous objects with high precision.}
            \resumeItem{Data Analysis using Python}
            {Analyzed laboratory data by making and fitting various plots using Python.}
            \resumeItem{Two-Photon Interferometry Simulation using Python}{Developed a \href{https://github.com/zhichen3/QA-sim}{\emph{simulation code}} that simulated the theoretical observational signal through two-photon interferometry given a set of telescopes and star pairs using Python.}
            \resumeItem{Additional Features: Data Sampling and MCMC}{Added features such as data sampling and Markov Chain Monte Carlo algorithm to predict the precision of the relative separation between the star pair.}
            % \resumeItem{Poster Conference}{Joined a poster conference by the end of the program to share research results over the summer.}
		\resumeItemListEnd
\vspace{-2pt}

    \resumeSubheading
        {SULI Program at Brookhaven National Lab}{}%{Remote}
        {Summer Internship}{Jun 2020 - Aug 2020}
        \resumeItemListStart
            \resumeItem{General Setting}
            {Worked in an Astro-group for building a radio-telescope to detect 21-cm emission from hydrogen}
            \resumeItem{Primary Role}{Developed additional features for \href{https://github.com/radiohep/imcurio} {\emph{imcurio}}, which is a simulation code for 21 cm intensity mapping observations that simulates a fixed set of visibilities for a fixed telescope and fixed sky using Python.}
            % \resumeItem{Poster Conference}{Joined a poster conference by the end of the program to share research results over the summer.}
        \resumeItemListEnd
		
    \resumeSubHeadingListEnd    
\vspace{-2pt}




\begin{comment}
%-----------PROJECTS-----------------
\section{Computational Projects}
\resumeSubHeadingListStart
\resumeSubItem{Vison - multimedia search engine (NLP, Search Engine, Web Crawlers, Multimedia Processing)}{(Work in progress) Research oriented, open source, search engine for bringing reverse multimedia search to small \& mid scale enterprises. Tech: Python, NodeJS, Intel OpenVino Toolkit, Selenium, TensorFlow (October '18)}
\vspace{2pt}
\resumeSubHeadingListEnd
\vspace{-5pt}
\end{comment}



%-----------Publications------------------
\section{Publications}
\resumeSubHeadingListStart

\resumeSubItem{\textbf{Chen, Z.}, Zingale, M., \& Eiden, K. (2023)}{\it Sensitivity of He Flames in X-Ray Bursts to Nuclear Physics. doi: 10.3847/1538-4357/acec72}

\resumeSubItem{Clark, A., Johnson, E., \textbf{Chen, Z.}, Eiden, K., Willcox, D., Boyd, B., Cao, L., DeGrendele, C., \& Zingale, M.. (2023).}{\it pynucastro: A Python Library for Nuclear Astrophysics. doi: 10.3847/1538-4357/acbaff}
\vspace{2pt}

\resumeSubItem{\textbf{Chen, Z.}, Nomerotski, A., Slosar, A., Stankus, P., \& Vintskevich, S.. (2022)}{\it Astrometry in two-photon interferometry using Earth rotation fringe scan. doi: 10.1103/PhysRevD.107.023015}

\vspace{2pt}
\resumeSubItem{Keach, M., Bellavia, S., \textbf{Chen, Z.}, Crawford, J., Dolzhenko, D., Figueroa, E., … Vintskevich, S. (2022).}{\it Increasing baselines and precision of optical interferometers using two-photon interference. A. Mérand, S. Sallum, \& J. Sanchez-Bermudez, Optical and Infrared Interferometry and Imaging VIII. doi:10.1117/12.2632122}
\resumeSubHeadingListEnd
\vspace{-5pt}


%-----------Awards-----------------
\section{Conferences and Meetings}
\resumeSubHeadingListStart
\resumeSubItem{Poster Presentation at Astronum Meeting, 2023}{\it Sensitivity of He Flames in X-ray Bursts to Nuclear Physics}
\resumeSubItem{Contributed Talk at the main CeNAM Frontiers in Nuclear Astrophysics Meeting, 2023}{{\it Sensitivity of He Flames in X-ray Bursts to Nuclear Physics}}
\resumeSubItem{SULI Internship Presentation at Brookhaven National Laboratory, Summer 2021}{\it Two-Photon Interferometry Simulations}
\resumeSubItem{SULI Internship Presentation at Brookhaven National Laboratory, Summer 2020}{\it Hydrogen Intensity Mapping Simulator: interpolation improvements and radio point source compatibility with application to GaLactic and Extragalactic All-Sky MWA Survey (GLEAM) dataset}

\resumeSubHeadingListEnd

\section{Honors, Grants, Awards, and Achievements}
\begin{description}[font=$\bullet$]
\item {Di-Tian Prize (2023): Awarded for excellent research and travel funding.}
\item {Stony Brook University, Academic Dean's List, 2017-2021}
\item {NYS School Academic Excellence Scholarship}
\vspace{-5pt}
\end{description}

\vspace{-5pt}

\section{Skills Summary}
\resumeSubHeadingListStart
	\resumeSubItem{Languages}{~~~~~~Python, C++, Bash}
	\resumeSubItem{Tools}{~~~~~~~~~~~~~~GIT, Latex, Sphinx, Doxygen, yt}
	\resumeSubItem{Platforms}{~~~~~~~Linux, Web, Windows}

\resumeSubHeadingListEnd

\vspace{-5pt}

\end{document}
